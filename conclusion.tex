\section{Discussion and future work}
We presented a novel approach to combining task and motion planning in
belief space through the \emph{maximum likelihood observation}
determinization principle. Our approach enforces a clear separation
between task planning, motion planning, and inference and has the
ability to integrate with complex state estimation methods (which are
too expensive to include in the inner loop of a planning
algorithm). We evaluated our algorithm on several domains, where it
completed challenging sequences of perception and geometric reasoning
by task planning in an abstract representation of the problem.

The primary limitation we encountered in our experiments stems from
computational complexity in rejection sampling. This can involve a
large number of intersection queries and in environments with many
geometric constraints this can slow planning down considerably.

Another issue arises from belief distributions with many modes. This
is a challenge for \mld{} approximations in general but can be
exacerbated by our system. As an extreme example, consider a search
task where the prior distribution has 100 modes scattered around a
large room. Our system will pick one of these to be the maximum
likelihood observation and proceed to investigate it. This decision is
made with no regard to the cost of investigating that location and so
this setting can result in plans that are quite suboptimal. While this
is a challenging problem and minimizing execution cost is not our
goal, it is clear that a better solution is needed for \ibsp{} to be
practical in similar scenarios.

Future work will include experiments on a real PR2, investigations to
the performance of this method with a noisy transition model for the
environment, and methods to deal with high variance distributions
efficiently.
