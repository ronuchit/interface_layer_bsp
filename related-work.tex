\section{Related Work}
%% Our work builds on recent results in deterministic task and motion
%% planning. Early work in task and motion planning was embodied in the
%% aSyMov system~\cite{gravot2005asymov}, where a task plan was generated
%% and used to guide motion planning. However, the focus was on
%% determining a way to solve motion plans in parallel. Dornhege et
%% al. use semantic attachments to do task and motion planning with
%% classical planners~\cite{dornhege2012semantic}. This approach makes
%% use of task planners, but requires representation of discretized
%% locations within the high level. Havur et al. use local search to find
%% an optimal geometric configuration before discretizing the continuous
%% plane into non-uniform cells and using a hybrid planner to find a
%% feasible motion plan~\cite{havur2014geometric}. Lagriffoul et
%% al. parameterize symbolic actions with coarse geometric
%% representations of objects and grasp types. Possible geometric
%% instances are sampled from the symbolic plan and are pruned by
%% constraints to reduce geometric
%% backtracking~\cite{lagriffoul2014orientation}.

Knowledge space planning representations are one of the oldest ideas
in AI~\cite{mccarthy1968some}. Bonet and Geffner provide in-depth
experimentation and analysis of discrete deterministic partially
observed planning~\cite{bonet2011planning}.  They provide conditions
under which discrete formulations of partially observable planning
(with binary, factored beliefs) can be compiled to sound and complete
representations. The conditions for this are similar to conditions for
our completeness theorem; however, our algorithms are aimed at large
continuous problems for robotics while their work is aimed at
classical planning.

Our approach relies on the maximum likelihood determinization
introduced by Platt et al.~\cite{platt2010belief}. They frame problem
solving in belief space as an underactuated control problem and apply
continuous techniques. 

The BHPN planning and execution system is a task and motion planning
approach to handle uncertainty~\cite{kaelbling2013integrated}. Similar
to our approach, they construct task plans in belief space under
maximum likelihood determinization. Levihn et al. extend this system
to construct shorter plans at execution time by smart replanning and
reconsideration~\cite{levihn2013foresight}. They explicityly formulate
the regression of belief goals under Gaussian distributions and plan
with an exact represention ofe belief state dynamics. In contrast, we
use references to the belief so we can plan with arbitrary beliefs
representations.

Srivastava et al. formulate open world POMDPs with a probabilistic
program~\cite{srivastava2014first} -- they develop a generalization of
point-based value iteration to that setting. While both our method and
theirs can been viewed as solving very large POMDPs, their method's size stems from
the (unbounded) size of the world and complexity of corresponding
beliefs, while ours stems from uncertainty about continuous quantities.


Gashler et al. use a contingent planner to generate plans that react
to feedback from the world~\cite{gaschler2013kvp}. They generate
contingent plans and use external function calls during their planning
step to generate effects of actions. However, they explicitly
represent poses and belief updates in symbolic models for task
planning. We use pose and belief references to plan abstractly. We
refine and verify plans with the \emph{true} belief.

Nebel et al. use a three valued logic for the TidyUp project that
allows fluents to take the value
\emph{uncertain}~\cite{nebel13aaaiirs}. In their system, however, they
assume that uncertain items become known once the robot is close
enough and so do not explicitly plan sensing actions. In contrast, we
represent uncertainty directly in our representation and combine
reasoning about uncertainty with reasoning about maximum likelihood
states.

%% Our determinization-replan approach shares similarity with
%% determinization-replan approaches for probabilistic planning, such as
%% FF-replan~\cite{yoon2007ff}. Both leverage determinization in order to
%% obtain major performance gains from classical planners. The maximum
%% likelihood assumption is similar to the most likely transition
%% determinization from that literature, while our optimistic belief
%% updates are similar to an all outcomes determinization.

