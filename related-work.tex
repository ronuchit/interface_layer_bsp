\section{Related Work}
Our work builds on recent results in deterministic task and motion
planning. Early work in task and motion planning was embodied in the
aSyMov system~\cite{gravot2005asymov}, where a task plan was generated
and used to guide motion planning. However, the focus was on
determining a way to solve motion plans in parallel. Dornhege et
al. use semantic attachments to do task and motion planning with
classical planners~\cite{dornhege2012semantic}. This approach makes
use of task planners, but requires representation of discretized
locations within the high level. Havur at al. use local search to find an optimal geometric configuration before descritizing the continuous plane into non-uniform cells and using a hybrid planner to find some feasible motion plan \cite{havur2014geometric}. Lagriffoul et al. parameterize symbolic actions with coarse geometric representations of objects and grasp types. Possible geometric instances are sampled from the symbolic plan and are pruned by constraints to reduce geometric backtracking. \cite{lagriffoul2014orientation}.

Maximum likelihood determinization was introduced by Platt et
al.~\cite{platt2010belief}. They used this approach to frame problem
solving in belief space as an underactuated control problem. They use
LQR and transcription methods to find trajectories in belief space and
characterize scenarios where a replanning strategy is guaranteed to
succeed. Van Den Berg et al. use LQG to solve Gaussian belief space
planning problems that incorporate collision avoidance
constraints~\cite{van2012motion}. They are able to plan without
the maximum likelihood assumption and explore the approximations
introduced with maximum likelihood determinizations.

Our determinization-replan approach shares similarity with
determinization-replan approaches for probabilistic planning, such as
FF-replan~\cite{yoon2007ff}. Both leverage determinization in order to
obtain major performance gains from classical planners. The maximum
likelihood assumption is similar to the most likely transition
determinization from that literature, while our optimistic belief
updates are similar to an all outcomes determinization.

The idea of using knowledge space representations is an old idea in
artificial intelligence that dates back to McCarthy and
Hayes~\cite{mccarthy1968some}. Bonet and Geffner provide in-depth
experimentation and analysis of discrete deterministic partially
observed planning~\cite{bonet2011planning}.  They provide conditions
under which discrete formulations of partially observable planning
(with binary, factored beliefs) can be compiled to sound and complete
representations. The conditions for this are similar to conditions
for our completeness theorem; however, our algorithms are aimed at
large continuous problems for robotics while their work is aimed at
classical planning.

The BHPN planning and execution system is a task and motion planning
approach to handle uncertainty~\cite{kaelbling2013integrated}. Similar
to our approach, they construct task plans in belief space under
maximum likelihood determinization. Levihn et al. extend this system to
construct shorter plans at execution time by smart replanning and
reconsideration~\cite{levihn2013foresight}. They rely on an aggressive
hierarchical strategy that delays detailed planning and commits to
abstract plans before refining them and formulate regression of belief
goals under Gaussian distributions to exactly represent the belief
state dynamics in the task planning formulation. In contrast, we use a
shallow hierarchy and develop full plans before execution; additionally, we use
references to belief distributions and avoid tying our planning
representation to the belief distribution.

Srivastava et al. formulate open world POMDPs with a probabilistic
program~\cite{srivastava2014first} -- they develop a generalization of
point-based value iteration to that setting. While both our method and
theirs can been viewed as solving very large POMDPs, their method's size stems from
the (unbounded) size of the world and complexity of corresponding
beliefs, while ours stems from uncertainty about continuous quantities.

Gashler et al. use a contingent planner in order to generate plans
that react to feedback from the world~\cite{gaschler2013kvp}. Their
system makes use of the contingent planner of Petrick and Bacchus
which maintains a 3-valued logic for fluent
values~\cite{petrick2002knowledge}. They generate contingent plans and
use external function calls during their planning step to generate
effects of actions. Their volume approximation shares many
similarities with our negative regions representation. Their
approach discretizes the world into separate locations and requires a
full specification of belief effects at the logical level. We plan with
the true belief and world state through an interfaced approach
in the determinized problem.

Nebel et al. use a three valued logic for the TidyUp project that
allows fluents to take the value
\emph{uncertain}~\cite{nebel13aaaiirs}. In their system, however, they
assume that uncertain items become known once the robot is close
enough and so do not explicitly plan sensing actions. In contrast, we
represent uncertainty directly in our representation and combine
reasoning about uncertainty with reasoning about maximum likelihood
states.

