\section{Interfaced belief space planning}
In this section, we present our primary contribution: \ibsp, a modular
approach to task and motion planning in belief
space. \secref{sec-bsp-formulation} extends our symbolic
representation to include belief states. \secref{sec-optimism} details
an approach based on optimistic observations and a simplified belief
representation to compactly represent belief space dynamics.
\secref{sec-ibsp} describes the \ibsp{} algorithm in detail and provides conditions
under which it is complete.

%%PA: this entire section is very technical; it'd be really nice if
%%the intro (or otherwise here, but probably the intro) had a
%%motivating example that illustrates the representation / algorithm
%%in action, and illustrates its strengths; this would bring the
%%reader here with already most of the intuition about our approach;
%%right now the reader arrives with limited intuition, and is just
%%getting a lot of formal stuff to read here...; 

\subsection{Belief space planning formulation}
\label{sec-bsp-formulation}
In our full description of the problem, we need to allow for
probabilistic representations in belief space. We restrict ourselves
to problems where the uncertainty is confined to the continuous state
(e.g., %%PA: -> e.g.,,
we may not know $o_1$'s location, but we know its name and whether
we are holding it). The entities for our partially observed problems
fall into one of the following three categories:
\begin{tightlist}
\item named objects
\item continuous variables (either object attributes or observations)
\item distributions over continuous variables
\end{tightlist}
Our fluents will describe relations between these entities. The
initial state and goal are defined in the same manner. Our input
additionally specifies a conditional distribution over observations
given the state.

We assume that our operators are separated into \emph{actions}, which
alter the true state but do not get any information about it, and
\emph{observations}, which do not alter the true state but garner
information about it. Actions are deterministic in belief space
(even when the underlying dynamics are not) and follow a similar
parametrization to the deterministic case.

The source of non-determinism in our specification is confined to
observations. An observation operator is parametrized by the belief
about state variables that determine its conditional distribution. To
formalize the \mld{}, we include a fluent, \emph{MLObs}(obs, $p_1$,
$p_2$, \ldots), that is true iff obs is the maximum likelihood
observation under the state distribution specified by the $p_i$. We
include the observation as a parameter of the operator and include the
\emph{MLObs} fluent as a precondition. This provides us a deterministic
planning problem of the kind described in \secref{sec-formulation}.


It is tempting to attempt to apply the skolemization approaches from
\cite{srivastava2014combined} to this representation. Unfortunately,
that is unlikely to be successful in practice. The issue is that the
number of skolemized observation variables will typically be
exponential in the number of skolemized beliefs. For example, consider
the following observation operator for our pick-and-place domain:\\
\noindent\emph{Observe}(obs,~r\_bel,~$o_1$\_bel,~\ldots,~r\_bel',~$o_1$\_bel',~\ldots)
\begin{tightlist}
\item[\emph{pre:}] \emph{Rloc}(r\_bel) $\wedge$ \emph{Loc}($o_1$, $o_1$\_bel) $\wedge$ \ldots\\
$\wedge$ \emph{Preimage\_RLoc}(obs, r\_bel, r\_bel')\\
$\wedge$ \emph{Preimage\_At}(obs, $o_1$, $o_1$\_bel, $o_1$\_bel') $\wedge$\ldots\\
$\wedge$ \emph{MLObs}(obs, r\_bel, $o_1$\_bel, \ldots)
\item[\emph{eff:}] $\lnot$\emph{Rloc}(r\_bel) $\wedge$ $\lnot$\emph{Loc}($o_1$, $o_1$\_bel) $\wedge$ \ldots\\
\emph{Rloc}(r\_bel') $\wedge$ \emph{Loc}($o_1$, $o_1$\_bel') $\wedge$ \ldots
\end{tightlist}
Creating a skolem function for the \emph{MLObs} fluent will
require considering all combinations of the other beliefs. For small
problems this approach may be reasonable, but for larger problems this
direct abstraction is intractable.

\subsection{Optimism for observations}
\label{sec-optimism}
The insight we use to solve this issue is that the \mld{} has very
restricted dynamics for our problem class. An observation operator can
only concentrate belief at the maximum likelihood location for a
distribution. An action operator, on the other hand, could change the
maximum likelihood location of a distribution (and potentially
introduce variance into our beliefs). Thus, we reformulate our problem in terms of two types of fluents:
those which characterize the maximum likelihood state, and those
which characterize the certainty in our belief. The latter of the two
shares similarities with the BVLoc($\epsilon$) fluents from Kaelbling
and Lozano-P\'erez~\cite{kaelbling2013integrated}.

We will illustrate this simplification with the \emph{Pick}
operator. To simplify description, we represent objects in the frame
of the robot and assume that the robot's pose in its own frame is
known. It is possible to relax this assumption at the cost of
introducing more complex descriptions.

Using a full belief representation, we have the following
description:\\
\emph{BPick}(o, o\_pose\_bel, r\_pose, grasp)
\vspace{-2mm}
\begin{tightlist}
\item[\emph{pre:}] \emph{BLoc}(o, o\_pose\_bel) $\wedge$
  \emph{RLoc}(r\_pose) \\$\wedge$ \emph{BGraspPose}(o, o\_pose\_bel,
  r\_pose, grasp) \\$\wedge$ \emph{Holding}(Null, Null)
\item[\emph{eff}:] \emph{Holding}(o, grasp) $\wedge$
  $\lnot$\emph{BLoc}(o, o\_pose\_bel) \\$\wedge$
  $\lnot$\emph{Holding}(Null, Null) \\$\wedge$ $\forall$t, h, g
  $\lnot$\emph{BObstructs}(o, t, h, g)\\$\wedge$ $\forall$ o\_p,
  r\_p $\lnot$\emph{BOccludes}(o, o\_p, r\_p)
\end{tightlist}
\emph{BGraspPose} characterizes distributions over the location of o
such that the probability with which a grasp succeeds is higher than
some safety margin. The issue is that this is only a conditional
effect of observation actions, so a sound representation can never
plan to make this true.  

To rectify this issue we use the following property about an \mld{}
representation. If \emph{BGraspPose} is not true, then one of two
cases will hold: either 1) the maximum likelihood state doesn't contain a valid grasp pose or 2) the maximum likelihood state admits at least 1 valid grasp pose, but the distribution has too much variance to allow
a successful grasp with high probability. In the first case, we must
manipulate the state (e.g., move obstructions), while in the other we must observe it. We
thus replace belief fluents with maximum likelihood fluents and
\emph{uncertainty} fluents:\\
\emph{MLPick}(o, o\_pose\_bel, r\_pose, grasp)
\vspace{-1mm}
\begin{tightlist}
\item[\emph{pre:}] \emph{MLLoc}(o, o\_pose\_bel) $\wedge$
  \emph{RLoc}(r\_pose) \\$\wedge$ \emph{MLGraspPose}(o,
  o\_pose\_bel, r\_pose, grasp) \\$\wedge$
  $\lnot$\emph{UncertainGP}(o\_pose\_bel, r\_pose, grasp)\\$\wedge$
  \emph{Holding}(Null, Null)
\item[\emph{eff}:] \emph{Holding}(o, grasp) $\wedge$
  $\lnot$\emph{MLLoc}(o, o\_pose\_bel) \\$\wedge$
  $\lnot$\emph{Holding}(Null, Null) \\$\wedge$ $\forall$t, h, g
  $\lnot$\emph{MLObstructs}(o, t, h, g)\\ $\wedge$ $\forall$ o\_p,
  r\_p $\lnot$\emph{MLOccludes}(o, o\_p, r\_p)
\end{tightlist}
In order to achieve $\lnot$\emph{UncertainGP} facts, we include an
observation operator:\\
\emph{ObserveGP}(r\_obs\_pose, o, o\_pose\_bel)
\begin{tightlist}
  \item[\emph{pre}:] \emph{MLLoc}(o, o\_pose\_bel) $\wedge$
    \emph{RLoc}(r\_obs\_pose) \\$\wedge$ $\forall o_i
    \lnot$\emph{MLOccludes}($o_i$, o\_pose\_bel, r\_obs\_pose) \\ $\wedge$
    \emph{MLInView}(r\_obs\_pose, o\_pose\_bel)
  \item[\emph{eff}:] $\forall$ r\_p, g $\lnot$\emph{UncertainGP}(o\_pose\_bel, r\_p, g)
\end{tightlist}
This corresponds to a conditional effect of the original
\emph{Observe} operator if we try to include \emph{UncertainGP}
fluents in that representation: the true effect can only be guaranteed
with complete knowledge of the true belief. However,
in order to correct a high level plan, we must include these fluents as a
possibility. We call this an \emph{optimistic} observation because the actual
effect is not guaranteed, and we thus rely on the interface layer to discover
cases when it does not hold and correct our representation. Note that
this assumption of optimism is only made in the description for the high
level. The full algorithm plans with a complete belief representation; if
multiple observations are needed to successfully localize an object, the algorithm
will plan to do them. In the event that our plan was overly optimistic BCheckSuccess on line 29 of \algoref{alg-ibsp} (described in the next section) will detect the overly optimistic behavior and will trigger a replan from the current state. Line 30 can be considered as recursive replanning.

\subsection{The \ibsp{} algorithm}
\label{sec-ibsp}
In this section, we detail the necessary changes to the interface from
Srivastava et al. to handle our belief space formulation. The refinement operator from Srivastava et al. didn't work well with belief states because it would exhaustively search all pose combinations before realizing the true failure was actually from a belief fluent (e.g., UncertainGP). The primary
change is that the refinement operation must find an assignment of
continuous variables that maximizes the probability of
`success.' \algoref{alg-ibsp} shows pseudocode with changes to \algoref{alg-ifop} highlighted in red.

\begin{algorithm}
 \caption{The \ibsp{} planning algorithm} \label{alg-ibsp}
 \begin{algorithmic}[1]
  \Procedure{IBSP}{$B_0, \gamma, \epsilon$}
  \State $B \leftarrow B_0$
  \State $b \leftarrow $ ExtractSymbols$(B_0, None)$
  \State $p_{post} \leftarrow $ Plan$(b, \gamma)$
  \State $p_{pre} \leftarrow None$
  \State $R_{pre} \leftarrow None$
  \While {not resource limit reached} 
     \State $R_{post} \leftarrow $ \textcolor{red}{MLRefine}$(B, p_{post})$
     \State $is\_fail,i_{fail}, failure$\\\ \ \ \ \ \ \ \ \ \ $\leftarrow $ \textcolor{red}{BCheckSuccess}$(B, (p_{post}, R_{post}), \epsilon)$
     \If {$\lnot is\_fail$}
         \State \Return $(p_{pre} + p_{post}, R_{pre} + R_{post})$
     \EndIf
     \State $p_{pre} \leftarrow p_{pre} + p_{post}[:i_{fail}]$
     \State $R_{pre} \leftarrow R_{pre} + R[:i_{fail}]$
     \State $B \leftarrow $ \textcolor{red}{MLFilter} $(B, p_{pre}, R_{pre})$
     \State $b\leftarrow $ ExtractFailureSymbols$(B, failure)$
     \State $p_{post} \leftarrow $ Plan$(b, \gamma)$     
     \If {max attempts reached}
        \State reset pose generators
        \State reset $B, b, p_{post}, p_{pre}, R_{pre}$
     \EndIf
  \EndWhile
  \EndProcedure
  \Procedure{IBSP-execute}{$B, \gamma, world, \epsilon$}
    \State $(p, R) \leftarrow IBSP(B, \gamma, \epsilon)$
    \For {$i \in len(p)$}
      \State $B \leftarrow$ ExecuteAndFilter$(p[i], R[i], B, world)$
      \If{$\lnot$BCheckSuccess$(B, (p[i+1:], R[i+1:]))$}
        \State \Return IBSP-execute$(B, \gamma, world, \epsilon)$
      \EndIf
    \EndFor
  \EndProcedure
 \end{algorithmic}
\end{algorithm}


We take as input a safety parameter $\epsilon$. This determines the
acceptable probability of failure for each step in the plan. The first
change is that we replace \emph{MotionPlan} with \emph{MLRefine}. The
difference is that \emph{MotionPlan} does collision-free motion
planning to connect successive states, while \emph{MLRefine} minimizes
the probability of collision. Both functions return a trajectory. \emph{MLRefine} can be defined either by
direct trajectory optimization that minimizes an unnormalized likelihood of collision,
or by sampling objects from the belief state and minimizing the number
of samples that are in collision.

The second change is the introduction of \emph{BCheckSuccess}, which examines a refinement and
checks that the preconditions
are satisfied. We assume that preconditions are defined by a function
that will return true or false indicating whether the precondition holds or not,
given a state of the world. We determine
that each of these is satisfied with probability greater than
$1-\epsilon$ by Monte Carlo sampling from the belief
state. $N_{samples}$ determines the fidelity of this sampling.

The third change is the addition of \emph{MLFilter} which updates the belief state based on the maximum likelihood observations that occurred.

When we detect failure for a belief fluent, we must decide which type
of failure information to propagate to the high level. We do this by
checking if the predicate holds in the maximum likelihood state. If it
does, then our error is caused by uncertainty in our belief state, and so
we propagate an uncertainty failure. Otherwise, we return a maximum
likelihood fluent as the violated precondition. The exceptions to this
rule are predicates about observations (e.g., occlusions). Because
these are deterministic in our representation, it does not make sense
to propagate uncertainty failures in this case. \algoref{alg-bcheck}
illustrates these ideas.


\begin{algorithm}
 \caption{Determining failure or success of a refinement} \label{alg-bcheck}
 \begin{algorithmic}[1]
  \Procedure{BCheckSuccess}{$B, p, R, \epsilon$}
      \For{$p_i , R_i$}
          \State preds $\leftarrow $ ExtractPreconds($p_i, R_i$)
          \State W $\leftarrow$ Sample(B, $N_{samples}$)
          \For{pred $\in$ preds}
              \If{ PercentSucceeds(pred, W) $< 1-\epsilon$}
                 \If {$\lnot$pred.eval(B.ML) or IsObs(pred)}
                     \State \Return (False, i, $\lnot$ML(pred))
                 \Else
                     \State \Return (False, i, Uncertain(pred))
                 \EndIf
              \EndIf
          \EndFor
      \EndFor
      \State \Return (True, -1, None)
  \EndProcedure
 \end{algorithmic}
\end{algorithm}
An important observation is that these implementations only require
us to be able to sample from our belief distribution and/or compute
unnormalized likelihoods. In order to compute the \mld{}, we need to be
able to compute a maximum likelihood state estimate and maintain a
filtered distribution. These are generic operations for belief states
and are typically implemented for most state estimation
techniques. This lightweight dependence on the form of the belief
distribution allows one to switch state estimators with no
change to the algorithm. Thus, our result is a belief space planning
algorithm that operates with only black box access to the belief
distribution. As an example, our experiments run with no modification
to the planning code for uniform distributions, unimodal Gaussian
distributions, and multimodal Gaussian distributions. Our implementation also works with discrete distributions, which can be formulated as a multimodal Gaussian where every Gaussian is collapsed to a Dirac delta function.

We can extend the completeness proof from Srivastava et
al.~\cite{srivastava2014combined} to provide conditions that the
\mld{} for a planning problem will be solved by \ibsp{}.
\begin{thm}
Let $P$ be a partially observable planning problem with deterministic
transitions. If the fully observed version of $P$ satisfies \emph{InterfacePlan}'s
conditions for completeness, then \algoref{alg-ibsp}
will find a solution to the \mld{} of $P$ with error
rate $\epsilon$ as $N_{samples}\rightarrow \infty$, provided that all
the calls to motion planning return.
\end{thm}
\begin{proof} (sketch) 
Deterministic state transitions imply that the domain is uniform with
respect to uncertainty predicates. The remaining predicates
characterize the fully observed problem and satisfy the conditions for
completeness by assumption. In the limit of infinite samples, we are able to
exactly evaluate whether predicates are satisfied. These combine to satisfy
the conditions to apply Theorem 1 from \cite{srivastava2014combined}.
\end{proof}
